%2multibyte Version: 5.50.0.2890 CodePage: 936
\documentclass[9pt]{article}%
\usepackage{bbm}
\usepackage{epsfig}
\usepackage{graphics,graphicx,amssymb,amsmath,verbatim}
\usepackage{amssymb}
\usepackage{mathrsfs}
\usepackage{amsfonts}
\usepackage{amsmath}
\usepackage{graphicx}
\usepackage{easybmat}
\usepackage{color}%
\setcounter{MaxMatrixCols}{30}
%TCIDATA{OutputFilter=latex2.dll}
%TCIDATA{Version=5.50.0.2890}
%TCIDATA{Codepage=936}
%TCIDATA{LastRevised=Tuesday, June 16, 2015 09:38:25}
%TCIDATA{<META NAME="GraphicsSave" CONTENT="32">}
%TCIDATA{<META NAME="SaveForMode" CONTENT="1">}
%TCIDATA{BibliographyScheme=Manual}
%TCIDATA{Language=American English}
%TCIDATA{ComputeDefs=
%$\dot{V}_{2}\left(  e\left(  t\right)  \right)  $
%$\gamma^{{}}$
%$N$
%$P=\left[
%\begin{array}
%[c]{cc}%
%1 & \\
%& t
%\end{array}
%\right]  $
%}
%BeginMSIPreambleData
\providecommand{\U}[1]{\protect \rule{.1in}{.1in}}
%EndMSIPreambleData
\setlength{\marginparwidth}{0.5in} \setlength{\marginparsep}{0.5in}
\setlength{\oddsidemargin}{-0in} \setlength{\evensidemargin}{-0in}
\setlength{\textwidth}{6.5in} \setlength{\topmargin}{-0.75in}
\setlength{\textheight}{9.5in} \setlength{\parindent}{0in}
\setlength{\parskip}{.1in}
\newtheorem{theorem}{Theorem}
\newtheorem{acknowledgment}{Acknowledgment}
\newtheorem{algorithm}{Algorithm}
\newtheorem{axiom}{Axiom}
\newtheorem{case}{Case}
\newtheorem{assumption}{Assumption}
\newtheorem{claim}{Claim}
\newtheorem{conclusion}{Conclusion}
\newtheorem{condition}{Condition}
\newtheorem{conjecture}{Conjecture}
\newtheorem{corollary}{Corollary}
\newtheorem{criterion}{Criterion}
\newtheorem{definition}{Definition}
\newtheorem{example}{Example}
\newtheorem{exercise}{Exercise}
\newtheorem{lemma}{Lemma}
\newtheorem{notation}{Notation}
\newtheorem{problem}{Problem}
\newtheorem{proposition}{Proposition}
\newtheorem{remark}{Remark}
\newtheorem{solution}{Solution}
\newtheorem{summary}{Summary}
\newenvironment{proof}[1][Proof]{\noindent \textbf{#1.} }{\  \rule{0.5em}{0.5em}}
\definecolor{blue}{rgb}{0,0,1}
\allowdisplaybreaks[4]
\begin{document}

\title{Regional Stabilization of Time-Delay Systems with Actuator Saturation and
Delay }
\author{Bin Zhou\thanks{The authors are with the Center for Control Theory and
Guidance Technology, Harbin Institute of Technology, Harbin, 150001, China.
Email: \texttt{binzhoulee@163.com, binzhou@hit.edu.cn}.}\quad \quad Xuefei
Yang$^{\ast}$}
\date{}
\maketitle

\begin{abstract}
to be added

\vspace{0.3cm}

\textbf{Keywords:} to be added

\end{abstract}

\section{Introduction}

\section{Problem Formulation}

In this section, we first consider the following linear time-delay system with
actuator saturation:%
\begin{equation}
\left \{
\begin{array}
[c]{l}%
\dot{x}\left(  t\right)  =Ax\left(  t\right)  +A_{\mathrm{r}}x\left(
t-r\right)  +B\mathrm{sat}\left(  u\left(  t\right)  \right) \\
x\left(  t\right)  =\phi \left(  t\right)  ,u\left(  t\right)  =\omega \left(
t\right)  ,\forall t\in \left[  -r,0\right]  ,
\end{array}
\right.  \label{tds}%
\end{equation}
where $r$ is nonnegative scalar, and $A,A_{\mathrm{r}}$ and $B$ are known
matrices. The function $\mathrm{sat}\left(  u\right)  $: $\mathbf{R}%
^{m}\rightarrow \mathbf{R}^{m}$ is the standard saturation defined as follows:%
\[
\mathrm{sat}\left(  u\left(  t\right)  \right)  =\left[
\begin{array}
[c]{ccc}%
\mathrm{sat}\left(  u_{1}\right)  & \cdots & \mathrm{sat}\left(  u_{2}\right)
\end{array}
\right]  ^{\mathrm{T}},
\]
with $\mathrm{sat}\left(  u_{i}\right)  =\mathrm{sign}\left(  u_{i}\right)
\min \left \{  1,\left \vert u_{i}\right \vert \right \}  .$ Let $\mathcal{C}%
_{n,r}=$ $\mathcal{C}\left(  \left[  -r,0\right]  ,\mathbf{R}^{n}\right)  $
denote the Banach space of continuous vector functions mapping the interval
$\left[  -r,0\right]  $ into $\mathbf{R}^{n}$ with the topology of uniform
convergence, and $x_{t}\in \mathcal{C}_{n,r}$ denote the restriction of
$x\left(  t\right)  $ to the interval $\left[  t-r,t\right]  $ translated to
$\left[  -r,0\right]  .$ When a state feedback $u\left(  t\right)
=F^{\mathrm{T}}x\left(  t\right)  $ is applied on to the system (\ref{tds}),
the resulting closed-loop system reads%
\begin{equation}
\left \{
\begin{array}
[c]{l}%
\dot{x}\left(  t\right)  =Ax\left(  t\right)  +A_{\mathrm{r}}x\left(
t-r\right)  +B\mathrm{sat}\left(  F^{\mathrm{T}}x\left(  t\right)  \right) \\
x\left(  t\right)  =\phi \left(  t\right)  ,\forall t\in \left[  -r,0\right]  .
\end{array}
\right.  \label{ctds}%
\end{equation}
For an initial condition $x_{0}\in \mathcal{C}_{n+m,r},$ denote the solution of
system (\ref{ctds}) as $x\left(  t,x_{0}\right)  .$ Assume that the trivial
solution $x\left(  t,x_{0}\right)  \equiv0$ is asymptotically stable, then the
domain of attraction of the origin is defined as%
\[
\mathcal{G}=\left \{  x_{0}\in \mathcal{C}_{n,r}:\underset{t\rightarrow \infty
}{\lim}\left \Vert x\left(  t,x_{0}\right)  \right \Vert =0\right \}  .
\]
A set $\mathcal{S}\subset \mathcal{C}_{n,r}$ is said to be invariant for system
(\ref{ctds}) if $x_{0}\in \mathcal{S}\Rightarrow x\left(  t,x_{0}\right)
\in \mathcal{S},\forall t\geq0.$ Moreover, a set $\mathcal{S}$ is called a
contractively invariant set if it is an invariant set and is in the domain of attraction.

\begin{problem}
Given a state gain $F$, how to determine whether the closed-loop system $y$
(\ref{ctds1}) is locally stable, and then to determine the local stability of
the system $x$ (\ref{ctds})?
\end{problem}

\begin{problem}
For a given $F,$ if system (\ref{ctds1}) is locally stable, how to find a set
$\mathcal{L}_{y}\subset \mathcal{\bar{C}}_{n,r}$ such that $\mathcal{L}_{y}$ is
an estimate of the domian of attraciton for the system (\ref{ctds1}), and is
as large as possible? As a result, to get a corresponding estimate of the
domian of attraciton $\mathcal{L}_{x}\subset \mathcal{C}_{n+m,r}$ for the
system (\ref{ctds}).
\end{problem}

\begin{problem}
How to disign an $F$ such that an estimate of the domain of attraction
$\mathcal{L}_{y}$ for $y$ system is maximized? And it follows that the
corresponding estimate of the domian of attraciton $\mathcal{L}_{x}$ for the
system (\ref{ctds}) is also maximized.
\end{problem}

Let $\mathcal{V}$ be the set of all $m\times m$ diagonal matrices whose
diagonal elements are either 1 or 0, then there are 2$^{m}$ elements in
$\mathcal{V}.$ Assume that each element in $\mathcal{V}$ is labeled as
$D_{i},i\in \mathcal{V}_{m}\triangleq \left \{  1,2,\ldots,2^{m}\right \}  $ and
let $D_{i}^{-}=I-D_{i},i\in \mathcal{V}_{m}.$ Then we give the following lemma
for later use.

\begin{lemma}
\label{lemma1}For two vectors $u,v\in \mathbf{R}^{m},$ if $\left \Vert
v\right \Vert _{\infty}\leq1,$ then%
\[
\mathrm{sat}\left(  u\right)  \in \mathrm{co}\left \{  D_{i}u+D_{i}^{-}%
v,i\in \mathcal{V}_{m}\right \}  ,
\]
where $\mathrm{co}\left \{  \cdot \right \}  $ denotes the convex hull of a set.
\end{lemma}

\begin{lemma}
\label{lemma2}For a given positive-define matrice $R\in \mathbf{R}^{n\times
n},$ any differentiable function $\omega$ in $\left[  a,b\right]
\longrightarrow \mathbf{R}^{n},$ the following inequality holds:%
\begin{equation}
\int_{a}^{b}\dot{\omega}^{\mathrm{T}}\left(  s\right)  R\dot{\omega
}^{\mathrm{T}}\left(  s\right)  \mathrm{d}s\geq \frac{1}{b-a}\left(
\omega \left(  b\right)  -\omega \left(  a\right)  \right)  ^{\mathrm{T}%
}R\left(  \omega \left(  b\right)  -\omega \left(  a\right)  \right)  .
\label{eq17}%
\end{equation}

\end{lemma}

\begin{lemma}
\label{lemma3}Let $x_{i}\in \mathbf{R}^{n},i\in \mathbf{I}\left[  1,m\right]
,m\geq1,$ be a series of vectors and $P>0$ be given. Then%
\begin{equation}
\left(  \underset{i1}{\overset{m}{\sum}}x_{i}\right)  ^{\mathrm{T}}P\left(
\underset{i1}{\overset{m}{\sum}}x_{i}\right)  \leq m\left(  \underset
{i1}{\overset{m}{\sum}}x_{i}^{\mathrm{T}}Px_{i}\right)  . \label{eq27}%
\end{equation}

\end{lemma}

\begin{lemma}
\label{lemma4}Let $\mathcal{A},\mathcal{B},\mathcal{H},\mathcal{P}$ and
$\mathcal{F}$ be real matrices of appropriate dimensions such that
$\mathcal{P}>0$ and $\mathcal{F}^{\mathrm{T}}\mathcal{F}\leq I$. Then for any
$\varepsilon>0$ such that $\mathcal{P}^{-1}-\varepsilon^{-1}\mathcal{BB}%
^{\mathrm{T}}>0,$%
\begin{equation}
\left(  \mathcal{A+BFH}\right)  ^{\mathrm{T}}\mathcal{P}\left(
\mathcal{A+BFH}\right)  \leq \mathcal{A}^{\mathrm{T}}\left(  \mathcal{P}%
^{-1}-\varepsilon^{-1}\mathcal{BB}^{\mathrm{T}}\right)  ^{-1}\mathcal{A+}%
\varepsilon \mathcal{H}^{\mathrm{T}}\mathcal{H}\text{.} \label{eq18}%
\end{equation}

\end{lemma}

\section{Main Results}

For $x_{t}\in \mathcal{C}_{n,r}$ and given positive-define matrices
$P\in \mathbf{R}^{n\times n},Q\in \mathbf{R}^{Nn\times Nn},R\in \mathbf{R}%
^{n\times n},$ and an integer $N\geq1,$ we choose a Laypunov-Krasovskii
functional candidate as%
\begin{equation}
V\left(  x_{t}\right)  =V_{1}\left(  x_{t}\right)  +V_{2}\left(  x_{t}\right)
+V_{3}\left(  x_{t}\right)  \label{eq8}%
\end{equation}
where%
\begin{align*}
V_{1}\left(  x_{t}\right)   &  =x^{\mathrm{T}}\left(  t\right)  Px\left(
t\right)  ,\\
V_{2}\left(  x_{t}\right)   &  =\int_{t-\frac{r}{N}}^{t}\pi^{\mathrm{T}%
}\left(  s\right)  Q\pi \left(  s\right)  \mathrm{d}s,\\
V_{3}\left(  x_{t}\right)   &  =\int_{-r}^{0}\int_{t+s}^{t}\dot{x}%
^{\mathrm{T}}\left(  \theta \right)  R\dot{x}\left(  \theta \right)
\mathrm{d}\theta \mathrm{d}s,
\end{align*}
and%
\[
\pi \left(  s\right)  =\left[
\begin{array}
[c]{c}%
x\left(  s\right) \\
x\left(  s-\frac{1}{N}r\right) \\
\vdots \\
x\left(  s-\frac{N-1}{N}\right)
\end{array}
\right]  .
\]
Then we can present the following theorem.

\begin{theorem}
Consider the linear time-delay system (\ref{ctds1}). Given $F\in
\mathbf{R}^{n\times m},r\neq0$ and $\rho>0.$ If there exist positive-define
matrices $P\in \mathbf{R}^{n\times n},Q\in \mathbf{R}^{Nn\times Nn}%
,R\in \mathbf{R}^{n\times n},$and $X_{\mathrm{r}}=\left[
\begin{array}
[c]{cccc}%
X_{\mathrm{r1}}^{\mathrm{T}} & X_{\mathrm{r2}}^{\mathrm{T}} & \cdots &
X_{\mathrm{r}\left(  N+2\right)  }^{\mathrm{T}}%
\end{array}
\right]  ^{\mathrm{T}}\in \mathbf{R}^{\left(  N+2\right)  n\times n}%
,H\in \mathbf{R}^{n\times m},$ non-singular $Y\in \mathbf{R}^{n\times n}$ such
that the following hold
\begin{equation}
\left[
\begin{array}
[c]{cc}%
\varpi \left(  i\right)   & \mathcal{X}_{\mathrm{r}}\\
\ast & -\frac{1}{r}\mathcal{R}%
\end{array}
\right]  <0,i\in \mathcal{V}_{m},\label{eq33}%
\end{equation}
and the relations%
\[
\left.
\begin{array}
[c]{c}%
\left \vert H_{i}^{\mathrm{T}}y\left(  t\right)  \right \vert \leq1,i\in \left \{
1,m\right \}  ,\\
H=\left[
\begin{array}
[c]{cccc}%
H_{1} & H_{2} & \cdots & H_{m}%
\end{array}
\right]  ,
\end{array}
\right.
\]
are satisfied for all $x_{t}\in \mathcal{L}_{V},$ where%
\[
\mathcal{L}_{V}=\left \{  \psi \in \mathcal{\bar{C}}_{n,r}:V\left(  \psi \right)
\leq \rho \right \}  ,
\]
and%
\begin{align*}
\varpi \left(  i\right)   &  =\Gamma_{\mathrm{P}i}^{\mathrm{T}}\left[
\begin{array}
[c]{cc}%
0 & \mathcal{P}\\
\mathcal{P} & 0
\end{array}
\right]  \Gamma_{\mathrm{P}i}+\Gamma_{\mathrm{R}i}^{\mathrm{T}}\left[
\begin{array}
[c]{cc}%
0 & \mathcal{R}\\
\mathcal{R} & 0
\end{array}
\right]  \Gamma_{\mathrm{R}i}+\Gamma_{\mathrm{Q}i}^{\mathrm{T}}\left[
\begin{array}
[c]{cc}%
0 & \mathcal{Q}\\
\mathcal{Q} & 0
\end{array}
\right]  \Gamma_{\mathrm{Q}i}\\
&  +\mathrm{He}\left(  \left[
\begin{array}
[c]{c}%
\mathcal{X}_{\mathrm{r1}}\\
\mathcal{X}_{\mathrm{r2}}\\
\vdots \\
\mathcal{X}_{\mathrm{r}\left(  N+2\right)  }%
\end{array}
\right]  \left[
\begin{array}
[c]{cccc}%
I_{n} & 0_{n,\left(  N-1\right)  n} & I_{n} & 0_{n,n}%
\end{array}
\right]  \right)  \\
&  +\mathrm{He}\left(  \left[
\begin{array}
[c]{c}%
I_{n}\\
I_{n}\\
\vdots \\
I_{n}%
\end{array}
\right]  \left[
\begin{array}
[c]{cccc}%
A & 0_{n,\left(  N-1\right)  n} & A_{\mathrm{r}} & -I_{n}%
\end{array}
\right]  \mathcal{Y}^{\mathrm{T}}\right)  \\
&  +\mathrm{He}\left(  \left[
\begin{array}
[c]{c}%
I_{n}\\
I_{n}\\
\vdots \\
I_{n}%
\end{array}
\right]  \left[
\begin{array}
[c]{cc}%
BD_{i}F^{\mathrm{T}}\mathfrak{Y}^{\mathrm{T}}+BD_{i}^{-}\mathcal{H} &
0_{n,\left(  N+1\right)  n}%
\end{array}
\right]  \right)  ,
\end{align*}


with%
\begin{align}
\Gamma_{\mathrm{P}i}  &  =\left[
\begin{array}
[c]{ccc}%
0_{n,n} & 0_{n,Nn} & I_{n}\\
I_{n} & 0_{n,Nn} & 0_{n,n}%
\end{array}
\right]  ,\Gamma_{\mathrm{R}i}=\left[
\begin{array}
[c]{ccc}%
0_{n,n} & 0_{n,Nn} & 0_{n,n}\\
0_{n,n} & 0_{n,Nn} & \sqrt{r}I_{n}%
\end{array}
\right]  ,\Gamma_{\mathrm{Q}i}=\left[
\begin{array}
[c]{ccc}%
I_{Nn} & 0_{Nn,n} & 0_{Nn,n}\\
0_{Nn,n} & I_{Nn} & 0_{Nn,n}%
\end{array}
\right]  ,\label{eq29}\\
\mathfrak{Y}  &  \mathfrak{=}Y^{-1},\mathcal{Y}\mathcal{=}\mathrm{diag}%
\left \{  \mathfrak{Y,Y,\cdots,Y}\right \}  _{\left(  N+2\right)  n}%
,\mathcal{Y}_{1}\mathcal{=}\mathrm{diag}\left \{  \mathfrak{Y,Y,\cdots
,Y}\right \}  _{Nn},\nonumber \\
\mathcal{P}  &  \mathcal{=}\mathfrak{Y}P\mathfrak{Y}^{\mathrm{T}}%
,\mathcal{R=}\mathfrak{Y}R\mathfrak{Y}^{\mathrm{T}},\mathcal{Q}=\mathcal{Y}%
_{1}Q\mathcal{Y}_{1}^{\mathrm{T}},\mathcal{H}=\mathfrak{Y}H,\nonumber \\
\mathcal{X}_{\mathrm{r}}  &  =\left[
\begin{array}
[c]{cccc}%
\mathcal{X}_{\mathrm{r1}}^{\mathrm{T}} & \mathcal{X}_{\mathrm{r2}}%
^{\mathrm{T}} & \cdots & \mathcal{X}_{\mathrm{r}\left(  N+2\right)
}^{\mathrm{T}}%
\end{array}
\right]  ^{\mathrm{T}},\mathcal{X}_{\mathrm{r}j}=\mathfrak{Y}X_{\mathrm{r}%
j}\mathfrak{Y}^{\mathrm{T}},j\in \mathbf{I}\left[  1,N+2\right]  ,\nonumber
\end{align}
then the solution $x\left(  t\right)  \equiv0$ is asymptotically stable for
the closed-loop system (\ref{ctds1}) with the set $\mathcal{L}_{V}$ contained
in the domain of attraction.
\end{theorem}

\begin{proof}
We choose the Lyapunov-Krasovskii functional in the form of (\ref{eq8}). Then
the derivatives of $V_{i}\left(  x_{t}\right)  ,i=1,2,3,$ are given by%
\begin{equation}
\left.
\begin{array}
[c]{l}%
\dot{V}_{1}\left(  x_{t}\right)  =\dot{x}^{\mathrm{T}}\left(  t\right)
Px\left(  t\right)  +x^{\mathrm{T}}\left(  t\right)  P\dot{x}\left(  t\right)
,\\
\dot{V}_{2}\left(  x_{t}\right)  =\pi^{\mathrm{T}}\left(  t\right)
Q\pi \left(  t\right)  -\pi^{\mathrm{T}}\left(  t-\frac{r}{N}\right)
Q\pi \left(  t-\frac{r}{N}\right)  ,\\
\dot{V}_{3}\left(  x_{t}\right)  =r\dot{x}^{\mathrm{T}}\left(  t\right)
R\dot{x}\left(  t\right)  -\int_{-r}^{0}\dot{x}^{\mathrm{T}}\left(
t+s\right)  R\dot{x}\left(  t+s\right)  \mathrm{d}s.
\end{array}
\right.  \label{eq10}%
\end{equation}


By denoting $\eta^{\mathrm{T}}\left(  t\right)  =\left[
\begin{array}
[c]{ccc}%
\pi^{\mathrm{T}}\left(  t\right)  & x^{\mathrm{T}}\left(  t-r\right)  &
\dot{x}^{\mathrm{T}}\left(  t\right)
\end{array}
\right]  ^{\mathrm{T}},$ we have%
\begin{align}
\dot{V}\left(  x_{t}\right)   &  =\dot{V}_{1}\left(  x_{t}\right)  +\dot
{V}_{2}\left(  x_{t}\right)  +\dot{V}_{3}\left(  x_{t}\right) \nonumber \\
&  =\eta^{\mathrm{T}}\left(  t\right)  \Gamma_{1}\eta \left(  t\right)
-\int_{t-r}^{t}\dot{x}^{\mathrm{T}}\left(  s\right)  R\dot{x}\left(  s\right)
\mathrm{d}s, \label{eq19}%
\end{align}
where $\Gamma_{1}$ is given by%
\begin{equation}
\Gamma_{1}=\Gamma_{\mathrm{P}i}^{\mathrm{T}}\left[
\begin{array}
[c]{cc}%
0 & P\\
P & 0
\end{array}
\right]  \Gamma_{\mathrm{P}i}+\Gamma_{\mathrm{R}i}^{\mathrm{T}}\left[
\begin{array}
[c]{cc}%
R & 0\\
0 & R
\end{array}
\right]  \Gamma_{\mathrm{R}i}+\Gamma_{\mathrm{Q}i}^{\mathrm{T}}\left[
\begin{array}
[c]{cc}%
Q & 0\\
0 & -Q
\end{array}
\right]  \Gamma_{\mathrm{Q}i}, \label{eq30}%
\end{equation}
where $\Gamma_{\mathrm{P}i},\Gamma_{\mathrm{R}i},\Gamma_{\mathrm{Q}i}$ are in
the form of (??). By using the Newton-Leibniz formula and the system
(\ref{ctds}), we have the following two identities:%
\begin{equation}
\left.
\begin{array}
[c]{l}%
0=2\eta^{\mathrm{T}}\left(  t\right)  X_{\mathrm{r}}\left(  x\left(  t\right)
-x\left(  t-r\right)  -\int_{t-r}^{t}\dot{x}\left(  s\right)  \mathrm{d}%
s\right) \\
0=2\eta^{\mathrm{T}}\left(  t\right)  Y_{\eta}\left(  Ax\left(  t\right)
+A_{\mathrm{r}}x\left(  t-r\right)  +B\mathrm{sat}\left(  F^{\mathrm{T}%
}x\left(  t\right)  \right)  -\dot{x}\left(  t\right)  \right)  ,
\end{array}
\right.  \label{eq20}%
\end{equation}
where $Y_{\eta}$ $\in \mathbf{R}^{\left(  N+2\right)  n\times n}$ is defined as%
\begin{equation}
Y_{\eta}=\left[
\begin{array}
[c]{cccc}%
Y & Y & \cdots & Y
\end{array}
\right]  ^{\mathrm{T}}. \label{eq21}%
\end{equation}
Inserting (\ref{eq20}) into (\ref{eq19}) gives%
\begin{align}
\dot{V}\left(  x_{t}\right)  =  &  \eta^{\mathrm{T}}\left(  t\right)
\Gamma_{3}\eta \left(  t\right)  +2\eta^{\mathrm{T}}\left(  t\right)  Y_{\eta
}B\mathrm{sat}\left(  F^{\mathrm{T}}x\left(  t\right)  \right) \nonumber \\
&  -\int_{t-r}^{t}z^{\mathrm{T}}\left(  s\right)  R^{-1}z\left(  s\right)
\mathrm{d}s\nonumber \\
\leq &  \eta^{\mathrm{T}}\left(  t\right)  \Gamma_{3}\eta \left(  t\right)
+2\eta^{\mathrm{T}}\left(  t\right)  Y_{\eta}B\mathrm{sat}\left(
F^{\mathrm{T}}x\left(  t\right)  \right)  , \label{eq22}%
\end{align}
where%
\[
z\left(  s\right)  =X_{\mathrm{r}}^{\mathrm{T}}\eta \left(  t\right)  +R\dot
{x}\left(  s\right)  ,
\]
and%
\begin{equation}
\Gamma_{3}=\Gamma_{1}+\Gamma_{2}+rX_{\mathrm{r}}R^{-1}X_{\mathrm{r}%
}^{\mathrm{T}} \label{eq23}%
\end{equation}
with $\Gamma_{2}$ defined as%

\begin{equation}
\Gamma_{2}=\mathrm{He}\left(  X_{\mathrm{r}}\left[
\begin{array}
[c]{cccc}%
I_{n} & 0_{n,\left(  N-1\right)  n} & I_{n} & 0_{n,n}%
\end{array}
\right]  \right)  +\mathrm{He}\left(  Y_{\eta}\left[
\begin{array}
[c]{cccc}%
A & 0_{n,\left(  N-1\right)  n} & A_{\mathrm{r}} & -I_{n}%
\end{array}
\right]  \right)  \label{eq24}%
\end{equation}
From Lemma \ref{lemma1}, there exist a series of nonnegative number
$\lambda_{i},i\in \mathcal{V}_{m}$ such that $\  \ $\
\begin{equation}
\mathrm{sat}\left(  F^{\mathrm{T}}y\left(  t\right)  \right)  =\underset
{i\in \mathcal{V}_{m}}{\sum}\lambda_{i}\left(  D_{i}F^{\mathrm{T}}+D_{i}%
^{-}H^{\mathrm{T}}\right)  y\left(  t\right)  ,\text{ }\underset
{i\in \mathcal{V}_{m}}{\sum}\lambda_{i}=1. \label{eq9}%
\end{equation}
We conclude from (\ref{eq9}) that,
\begin{align}
2\eta^{\mathrm{T}}\left(  t\right)  Y_{\eta}B\mathrm{sat}\left(
F^{\mathrm{T}}x\left(  t\right)  \right)   &  =2\eta^{\mathrm{T}}\left(
t\right)  Y_{\eta}B\underset{i\in \mathcal{V}_{m}}{\sum}\lambda_{i}\left(
D_{i}F^{\mathrm{T}}+D_{i}^{-}H^{\mathrm{T}}\right)  x\left(  t\right)
\nonumber \\
&  =2\eta^{\mathrm{T}}\left(  t\right)  \underset{i\in \mathcal{V}_{m}}{\sum
}\lambda_{i}\left[
\begin{array}
[c]{cc}%
Y_{\eta}B\left(  D_{i}F^{\mathrm{T}}+D_{i}^{-}H^{\mathrm{T}}\right)  &
0_{n,\left(  N+1\right)  n}%
\end{array}
\right]  \eta \left(  t\right) \nonumber \\
&  \leq \eta^{\mathrm{T}}\left(  t\right)  \underset{i\in \mathcal{V}_{m}}{\max
}\left \{  \Gamma_{4}\left(  i\right)  \right \}  \eta \left(  t\right)  ,
\label{eq28}%
\end{align}
where%
\begin{equation}
\Gamma_{4}\left(  i\right)  =\mathrm{He}\left(  \left[
\begin{array}
[c]{cc}%
Y_{\eta}B\left(  D_{i}F^{\mathrm{T}}+D_{i}^{-}H^{\mathrm{T}}\right)  &
0_{n,\left(  N+1\right)  n}%
\end{array}
\right]  \right)  . \label{eq31}%
\end{equation}
Then substituting (\ref{eq28}) into (\ref{eq22}) gives%
\[
\dot{V}\left(  x_{t}\right)  \leq \underset{i\in \mathcal{V}_{m}}{\max}%
\eta^{\mathrm{T}}\left(  t\right)  \varsigma \left(  i\right)  \eta \left(
t\right)  ,
\]
where $\varsigma \left(  i\right)  =\Gamma_{1}+\Gamma_{2}+\Gamma_{4}\left(
i\right)  +rX_{\mathrm{r}}R^{-1}X_{\mathrm{r}}^{\mathrm{T}}.$

By the Schur complement, $\varsigma \left(  i\right)  <0,i\in \mathcal{V}_{m}$
is equivalent to%
\begin{equation}
\left[
\begin{array}
[c]{cc}%
\Gamma_{1}+\Gamma_{2}+\Gamma_{4}\left(  i\right)  & X_{\mathrm{r}}\\
\ast & -\frac{1}{r}R
\end{array}
\right]  <0, \label{eq32}%
\end{equation}
which is equivalent to (\ref{eq33}) by setting $\mathfrak{Y}\mathfrak{=}%
Y^{-1},\mathcal{Y}\mathcal{=}\mathrm{diag}\left \{  \mathfrak{Y,Y,\cdots
,Y}\right \}  _{\left(  N+2\right)  n},\mathcal{Y}_{1}\mathcal{=}%
\mathrm{diag}\left \{  \mathfrak{Y,Y,\cdots,Y}\right \}  _{Nn},$

$\mathcal{P}\mathcal{=}\mathfrak{Y}P\mathfrak{Y}^{\mathrm{T}},\mathcal{R=}%
\mathfrak{Y}R\mathfrak{Y}^{\mathrm{T}},\mathcal{Q}=\mathcal{Y}_{1}%
Q\mathcal{Y}_{1}^{\mathrm{T}},\mathcal{H}=\mathfrak{Y}H,\mathcal{X}%
_{\mathrm{r}j}=\mathfrak{Y}X_{\mathrm{r}j}\mathfrak{Y}^{\mathrm{T}}%
,j\in \mathbf{I}\left[  1,N+2\right]  $ and performing a congruence
transformation by $\mathcal{Y}$ to (\ref{eq32}). That is to say that
(\ref{eq32}) holds if (\ref{eq33}) is satisfied. Then we have $\dot{V}\left(
x_{t}\right)  <0.$
\end{proof}


\end{document}